\documentclass[lettersize,journal]{IEEEtran}
\usepackage{amsmath,amsfonts}
\usepackage{algorithmic}
\usepackage{algorithm}
\usepackage{array}
\usepackage[caption=false,font=normalsize,labelfont=sf,textfont=sf]{subfig}
\usepackage{textcomp}
\usepackage{stfloats}
\usepackage{url}
\usepackage{verbatim}
\usepackage{graphicx}
\usepackage{cite}
\hyphenation{}

\begin{document}

\title{\texttt{quadcopter} package}
\author{dinay-kingkiller}

\maketitle

%\begin{abstract}
%\end{abstract}

\section{Introduction}
\IEEEPARstart{T}{he} \texttt{quadcopter} package is filled with obtuse and often confusing code derived from various physics equations and mathematical formulas. This paper hopes to bridge the gap between quadcopter theory and the \texttt{src} files included. This paper can also be used as a reference for later expanding this package.

\section{Model}
\subsection{Rotations}
The rotation of 
\begin{eqnarray}
  _N\boldsymbol\omega_A = \dot{a} \mathbf{\hat{n_z}} = \dot{a} \mathbf{\hat{a_z}}
\end{eqnarray}
where $a$ is the \emph{yaw}, $b$ is the \emph{pitch}, $c$ is the \emph{roll}.


% \section*{Acknowledgments}



%{\appendix[Proof of Zonklar Equations]
%Use $\backslash${\tt{appendix}} if you have a single appendix:
% Do not use $\backslash${\tt{section}} anymore after $\backslash${\tt{appendix}}, only $\backslash${\tt{section*}}.
% If you have multiple appendixes use $\backslash${\tt{appendices}} then use $\backslash${\tt{section}} to start each appendix.
%You must declare a $\backslash${\tt{section}} before using any $\backslash${\tt{subsection}} or using $\backslash${\tt{label}} ($\backslash${\tt{appendices}} by itself starts a section numbered zero.)}



%{\appendices
%\section*{Proof of the First Zonklar Equation}
%Appendix one text goes here.
% You can choose not to have a title for an appendix if you want by leaving the argument blank
%\section*{Proof of the Second Zonklar Equation}
%Appendix two text goes here.}



% \begin{thebibliography}{1}
% \bibliographystyle{IEEEtran}


%\end{thebibliography}



% \vfill

\end{document}


