\documentclass[lettersize,journal]{IEEEtran}
\usepackage{amsmath,amsfonts}
\usepackage{algorithmic}
\usepackage{algorithm}
\usepackage{array}
\usepackage[caption=false,font=normalsize,labelfont=sf,textfont=sf]{subfig}
\usepackage{textcomp}
\usepackage{stfloats}
\usepackage{url}
\usepackage{verbatim}
\usepackage{graphicx}
\usepackage{cite}
\hyphenation{}

\begin{document}

\title{\texttt{quadcopter} package}
\author{dinay-kingkiller}

\maketitle

%\begin{abstract}
%\end{abstract}

\section{Introduction}
\IEEEPARstart{T}{he} \texttt{quadcopter} package is filled with obtuse and often confusing code derived from various physics equations and mathematical formulas. This paper hopes to bridge the gap between quadcopter theory and the \texttt{src} files included. This paper can also be used as a reference for later expansion of this package.

\section{The Model Node}
While an actual quadcopter would be helpful, a simulation is the next best thing. The model node simulates the physics of a quadcopter. The physics in this section also provide insight on how motor input and sensor output behave.

\subsection{Rotational Kinetics}

Quaternions are integral in understanding the implementation of the model node. Firstly, a vector in the inertial frame $N$ can be transformed and expressed in the robot frame $R$ by the quaternion $\mathrm{q}$:
\begin{equation}
	\mathbf{v}^R = \mathrm{q} \mathbf{v}^N \mathrm{q}^*
\end{equation}

That quaternion derivative can be updated via the angular velocity from robot frame to inertial frame $\boldsymbol\omega$ via
\begin{equation}
	\mathrm{\dot{q}} = \frac{1}{2} \boldsymbol\omega \mathrm{q}
\end{equation}

Integrating over the unit 4-sphere:
\begin{equation}
	\mathrm{q}[k+1] = \exp\left(\frac{1}{2}\boldsymbol\omega[k] \Delta t\right) \mathrm{q}[k]
\end{equation}

\begin{eqnarray}
	\mathrm{q}_s = \cos\left(\frac{\left|\boldsymbol\omega\right|\Delta t}{2}\right)\nonumber\\
	\mathrm{q}_v = \boldsymbol\omega \sin\left(\frac{\left|\boldsymbol\omega\|\Delta t}{2}\right) \nonumber
\end{eqnarray}

Throughout this paper, I use notation of the sum of scaled unit vectors (versors), but coding is better described as \texttt{float}s. The dot product of vectors provides these scalars. The rotation matrices provide a way to move between these \texttt{array}s of \texttt{float}s. For example, consider the rotation matrix from basis $A$ to $B$.
\begin{equation}
  \begin{pmatrix}
    \mathbf{v} \cdot \mathbf{\hat{b}}_x \\
    \mathbf{v} \cdot \mathbf{\hat{b}}_y \\
    \mathbf{v} \cdot \mathbf{\hat{b}}_z
  \end{pmatrix}
  =
  {^\mathrm{B}R^\mathrm{A}}
  \begin{pmatrix}
    \mathbf{v} \cdot \mathbf{\hat{a}}_x \\
    \mathbf{v} \cdot \mathbf{\hat{a}}_y \\
    \mathbf{v} \cdot \mathbf{\hat{a}}_z
  \end{pmatrix}
\end{equation}

The rotation matrices also collect the dot products to push final calculations down the road.
\begin{equation}
  {^\mathrm{B}R^\mathrm{A}} = 
  \begin{pmatrix}
    \mathbf{\hat{a}}_x \cdot \mathbf{\hat{b}}_x &
    \mathbf{\hat{a}}_y \cdot \mathbf{\hat{b}}_x &
    \mathbf{\hat{a}}_z \cdot \mathbf{\hat{b}}_x \\
    \mathbf{\hat{a}}_x \cdot \mathbf{\hat{b}}_y &
    \mathbf{\hat{a}}_y \cdot \mathbf{\hat{b}}_y &
    \mathbf{\hat{a}}_z \cdot \mathbf{\hat{b}}_y \\
    \mathbf{\hat{a}}_x \cdot \mathbf{\hat{b}}_z &
    \mathbf{\hat{a}}_y \cdot \mathbf{\hat{b}}_z &
    \mathbf{\hat{a}}_z \cdot \mathbf{\hat{b}}_z
  \end{pmatrix}
\end{equation}

The rotation of the quadcopter can be described by a composition of three rotations: from a Newtonian frame $N$ to $A$, by an angle $a$ around their common $x$ axis (the \emph{roll}); from intermediate frames $A$ to $B$, by angle $b$ around their common $y$ axis (the \emph{pitch}); and from $B$ to the quadcopter body frame $C$, by angle $c$ around their common $z$ axis (the \emph{yaw}). The angles and bases are chosen to make the angular velocities positive. On the quadcopter frame $\mathbf{\hat{c}}_x$ points to the front motor, $\mathbf{\hat{c}}_y$ points to the left motor, and $\mathbf{\hat{c}}_z$ is the cross product of the two.
\begin{eqnarray}
  {_\mathrm{N}\boldsymbol\omega_\mathrm{A}} =& \dot{a} \mathbf{\hat{n}}_x &= \dot{a} \mathbf{\hat{a}}_x \label{ARotVel}\\
  {_\mathrm{A}\boldsymbol\omega_\mathrm{B}} =& \dot{b} \mathbf{\hat{a}}_y &= \dot{b} \mathbf{\hat{b}}_y\\
  {_\mathrm{B}\boldsymbol\omega_\mathrm{C}} =& \dot{c} \mathbf{\hat{b}}_z &= \dot{c} \mathbf{\hat{c}}_z \label{CRotVel}
\end{eqnarray}

Using the notation of the rotation matrices above, the composition of rotation matrices is the matrix product.
\begin{equation}
  {^\mathrm{C}R^\mathrm{N}} = \left({^\mathrm{C}R^\mathrm{B}}\right)\left({^\mathrm{B}R^\mathrm{A}}\right)\left({^\mathrm{A}R^\mathrm{N}}\right)
\end{equation}

The important combinations of dot products can be found in tables \ref{InertialTransforms} and \ref{RobotTransforms}. 
\begin{table}[!t] \centering
  \caption{Transforms from the inertial frame $\mathrm{N}$\\($s_x = \sin x$, $c_x = \cos x$)}
  \begin{tabular}{c|ccc}
    $^\mathrm{U}R^\mathrm{N}$ & $\mathbf{\hat{n}}_x$ & $\mathbf{\hat{n}}_y$ & $\mathbf{\hat{n}}_z$ \\
    \hline
    $\mathbf{\hat{a}}_x$ & $1$ & $0$ & $0$\\
    $\mathbf{\hat{a}}_y$ & $0$ & $c_a$ & $s_a$\\
    $\mathbf{\hat{a}}_z$ & $0$ & $-s_a$ & $c_a$\\
    \hline
    $\mathbf{\hat{b}}_x$ & $c_b$ & $s_as_b$ & $-s_ac_a$\\
    $\mathbf{\hat{b}}_y$ & $0$ & $c_a$ & $s_a$\\
    $\mathbf{\hat{b}}_z$ & $s_b$ & $-s_a c_b$ & $c_a c_b$\\
    \hline
    $\mathbf{\hat{c}}_x$ & $c_b c_c$ & $-c_a s_c + s_a s_b c_c$ & $-s_a s_c - c_a s_b c_c$\\
    $\mathbf{\hat{c}}_y$ & $c_b s_c$ & $c_a c_c + s_a s_b s_c$ & $s_a c_c - c_a s_b s_c$\\
    $\mathbf{\hat{c}}_z$ & $s_b$ & $-s_a c_b$ & $c_a c_b$
  \end{tabular} \label{InertialTransforms}
\end{table}

\begin{table}[!t] \centering
  \caption{Transforms from the robot frame $\mathrm{C}$\\($s_x = \sin x$, $c_x = \cos x$)}
  \begin{tabular}{c|ccc}
    $^\mathrm{U}R^\mathrm{C}$ & $\mathbf{\hat{c}}_x$ & $\mathbf{\hat{c}}_y$ & $\mathbf{\hat{c}}_z$ \\
    \hline
    $\mathbf{\hat{b}}_x$ & $c_c$ & $s_c$ & $0$ \\
    $\mathbf{\hat{b}}_y$ & $-s_c$ & $c_c$ & $0$ \\
    $\mathbf{\hat{b}}_z$ & $0$ & $0$ & $1$ \\
    \hline
    $\mathbf{\hat{a}}_x$ & $c_b c_c$ & $c_b s_c$ & $s_b$ \\
    $\mathbf{\hat{a}}_y$ & $-s_c$ & $c_c$ & $0$ \\
    $\mathbf{\hat{a}}_z$ & $s_b c_c$ & $-s_b s_c$ & $c_b$ \\
    \hline
    $\mathbf{\hat{n}}_x$ & $c_b c_c$ & $c_b s_c$ & $s_b$ \\
    $\mathbf{\hat{n}}_y$ & $-c_a s_c + s_a s_b s_c$ & $c_a c_c + s_a s_b s_c$ & $-s_a c_b$ \\
    $\mathbf{\hat{n}}_z$ & $-s_a s_c-c_a s_b c_c$ & $s_a c_c - c_a s_b s_c$ & $c_a c_b$
  \end{tabular} \label{RobotTransforms}
\end{table}

\subsection{Linear Equations of Motion}
If you choose the right reference frames the linear equations of motion are easy to express. First, the motion of the quadcopter from its origin point is defined simply:
\begin{eqnarray}
  \mathbf{r}_{c/o} &=& x \mathbf{\hat{n}}_x + y \mathbf{\hat{n}}_y + z \mathbf{\hat{n}}_z \\
  {_\mathrm{N}\mathbf{v}_c} &=& \dot{x} \mathbf{\hat{n}}_x + \dot{y} \mathbf{\hat{n}}_y + \dot{z} \mathbf{\hat{n}}_z \\
  {_\mathrm{N}\mathbf{a}_c} &=& \ddot{x} \mathbf{\hat{n}}_x + \ddot{y} \mathbf{\hat{n}}_y + \ddot{z} \mathbf{\hat{n}}_z
\end{eqnarray}

For our model we assume each motor puts out a force proportional to its square. If we let $m$ be the total mass of the robot and $k$ be the proportionality constant:
\begin{equation}
  m{_\mathrm{N}\mathbf{a}_c} = k \left(\omega_f^2 + \omega_l^2 + \omega_b^2 + \omega_r^2\right) \mathbf{\hat{c}}_y - mg \mathbf{\hat{n}}_y
\end{equation}
Where $f$, $l$, $b$, and $r$ stand for the front, right, back and left motors respectively. Expressed in the inertial frame $N$ the equations of motion are:
\begin{eqnarray}
  \ddot{x} &=& \frac{k}{m} \Omega \cos b \sin c \\
  \ddot{y} &=& \frac{k}{m} \Omega \cos a \cos c + \frac{k}{m} \Omega \sin a \sin b \sin c - g\\
  \ddot{z} &=& \frac{k}{m} \Omega \sin a \cos c - \frac{k}{m} \Omega \cos a \sin b \sin c 
\end{eqnarray}
where $\Omega$ is the sum of the squares of the motor inputs.



\section{Rotational Momentum}
Supposing that most of the mass of the quadcopter is in the motors, the inertia dyadics for the four motors is:
\begin{eqnarray}
  \mathbf{I}^{f/c} &=& \frac{mL^2}{4} \mathbf{\hat{c}}_y\mathbf{\hat{c}}_y + \frac{mL^2}{4} \mathbf{\hat{c}}_z \mathbf{\hat{c}}_z \\
  \mathbf{I}^{l/c} &=& \frac{mL^2}{4} \mathbf{\hat{c}}_x\mathbf{\hat{c}}_x + \frac{mL^2}{4} \mathbf{\hat{c}}_z \mathbf{\hat{c}}_z \\
  \mathbf{I}^{b/c} &=& \frac{mL^2}{4} \mathbf{\hat{c}}_y\mathbf{\hat{c}}_y + \frac{mL^2}{4} \mathbf{\hat{c}}_z \mathbf{\hat{c}}_z \\
  \mathbf{I}^{r/c} &=& \frac{mL^2}{4} \mathbf{\hat{c}}_x\mathbf{\hat{c}}_x + \frac{mL^2}{4} \mathbf{\hat{c}}_z \mathbf{\hat{c}}_z
\end{eqnarray}

The total inertial dyadic of the robot is then
\begin{equation}
  \mathbf{I}^{R/c} = \frac{mL^2}{2} \left(\mathbf{\hat{c}}_x\mathbf{\hat{c}}_x + \mathbf{\hat{c}}_y\mathbf{\hat{c}}_y + 2 \mathbf{\hat{c}}_z\mathbf{\hat{c}}_z\right)
\end{equation}

The angular velocity is the compositions of simple velocities given in equations \ref{ARotVel}-\ref{CRotVel}.
\begin{equation}
  {_\mathrm{N}\boldsymbol\omega_C} = \dot{a}\mathbf{\hat{a}}_x + \dot{b} \mathbf{\hat{b}}_y + \dot{c} \mathbf{\hat{c}}_z
\end{equation}

And the angular momentum is the dot product of the two
\begin{eqnarray}
  {^\mathrm{N}\mathbf{H}^{R/c}} &=& \mathbf{I}^{R/c} \cdot {_\mathrm{N}\boldsymbol\omega_\mathrm{C}}\nonumber\\
  &=&\frac{mL^2}{4}\left(\dot{a} \cos b \cos c - \dot{b} \sin c\right) \mathbf{\hat{c}}_x\nonumber\\
  &&+\frac{mL^2}{4}\left(\dot{a} \cos b \sin c + \dot{b} \cos c\right) \mathbf{\hat{c}}_y\nonumber\\
  &&+\frac{mL^2}{4}\left(\dot{a}\sin b + 2\dot{c}\right) \mathbf{\hat{c}}_z
\end{eqnarray}


The angular acceleration (in the inertial frame) can be found with some manipulation of derivatives in reference frames.
\begin{eqnarray}
  {_\mathrm{N}\boldsymbol\alpha_\mathrm{C}} &=& \frac{^\mathrm{N}\mathrm{d}}{\mathrm{d}t} {_\mathrm{N}\boldsymbol\omega_\mathrm{C}} \nonumber\\
  &=& \frac{^\mathrm{N}\mathrm{d}}{\mathrm{d}t} \dot{a} \mathbf{\hat{n}}_x + \frac{^\mathrm{A}\mathrm{d}}{\mathrm{d}t} \dot{b} \mathbf{a}_y + {_\mathrm{N}\boldsymbol\omega_\mathrm{A}} \times \dot{b} \mathbf{\hat{a}}_y\nonumber\\
  && + \frac{^\mathrm{B}\mathrm{d}}{\mathrm{d} t} \dot{c} \mathbf{\hat{b}}_z + {_\mathrm{N}\boldsymbol\omega_\mathrm{B}} \times \dot{c} \mathbf{\hat{b}}_z \nonumber\\
  &=& \ddot{a}\mathbf{\hat{n}}_x + \ddot{b} \mathbf{\hat{a}}_y + \ddot{c} \mathbf{\hat{c}}_z\nonumber\\
  &&+\dot{a}\dot{b}\mathbf{\hat{a}}_z + \dot{b}\dot{c} \mathbf{\hat{b}}_x + \dot{a}\dot{c} \left(\mathbf{\hat{n}}_x \times \mathbf{\hat{c}}_z\right) 
\end{eqnarray}

The moment from the reference frame $\mathbf{M}^* = \mathbf{I}\cdot \boldsymbol\alpha + \boldsymbol\omega\times \mathbf{H}$ was calculated with a script
\begin{eqnarray}
  \mathbf{M}^* \cdot \mathbf{\hat{c}}_x &=& \ddot{a}\cos b\cos c + \ddot{b}\sin c- \dot{a}\dot{b}\sin b\cos c\nonumber\\
  &&- \dot{a}\dot{c}\sin c\cos b + \dot{b}\dot{c}\cos c\\
  \mathbf{M}^* \cdot \mathbf{\hat{c}}_y &=& - \ddot{a}\sin c\cos b + \ddot{b}\cos c+\dot{a}\dot{b}\sin b\sin c\nonumber\\
  &&- \dot{a}\dot{c}\cos b\cos c - \dot{b}\dot{c}\sin c \\
  \mathbf{M}^* \cdot \mathbf{\hat{c}}_z &=& \ddot{a} \sin b + \ddot{c} +\dot{a} \dot{b} \cos b
\end{eqnarray}



% \section*{Acknowledgments}

\appendices
\section*{Definition of a Quaternion}
Quadcopters rotate. And flip. Sometimes they even hover. In order to do any of these things, the rotation of the quadcopter must be tracked. A great way for tracking rotations is unit quaternions. This section covers the basic algebra the quaternion ring $\left(\mathbb{H}, +, \otimes\right)$.
First, a quaternion is a member of a 4-dimensional vector space. Addition is defined simply:
\begin{equation}
  \begin{pmatrix}
    p_0 \\
    p_1 \\
    p_2 \\
    p_3
  \end{pmatrix} +
  \begin{pmatrix}
    q_0 \\
    q_1 \\
    q_2 \\
    q_3
  \end{pmatrix} =
  \begin{pmatrix}
    p_0 + q_0 \\
    p_1 + q_1 \\
    p_2 + q_2 \\
    p_3 + q_3
  \end{pmatrix}
\end{equation}
Multiplication is more complex
\begin{equation}
 \begin{pmatrix}
    p_0 \\
    p_1 \\
    p_2 \\
    p_3
  \end{pmatrix} \otimes 
  \begin{pmatrix}
    q_0 \\
    q_1 \\
    q_2 \\
    q_3
  \end{pmatrix} =
  \begin{pmatrix}
    p_0q_0 - p_1q_1 - p_2q_2 - p_3q_3 \\
    p_0q_1 + p_1q_0 + p_2q_3 - p_3q_2 \\
    p_0q_2 - p_1q_3 + p_2q_0 + p_3q_1 \\
    p_0q_3 + p_1q_2 - p_2q_1 + p_3q_0
  \end{pmatrix}
\end{equation}
Please note, quaternion multiplication is not commutative (For most quaternions: $p\otimes q\neq q \otimes p$). Some other algebraic properties:
Every scalar can be expressed as a quaternion $(s, 0, 0, 0)$ and every (3-)vector can be a quaternion $(0, v1, v2, v3)$. The 0-quaternion $(0, 0, 0, 0)$ is the additive identity and the 1-quaternion the multiplicative identity $(1, 0, 0, 0)$. The conjugate of a quaternion has inverted signs on the vector components: $q^*=(q0, q1, q2, q3)^*=(q0, -q1, -q2, -q3)$. The norm of a quaternion is the square root of the components, or the square of the quaternion multiplied by its conjugate (if we treat a purely scalar quaternion as a scalar).
\begin{equation}
  \|q\| = \sqrt{q\otimes q^*} = \sqrt{q_0^2+q_1^2+q_2^2+q_3^2}
\end{equation}
Every quaternion has an inverse $q^{-1}$ such that $q\otimes q^{-1}=1=q^{-1}\otimes q$. The inverse is $q^{-1}=q^*/\|q\|$. A quaternion with norm equal to one is called a unit quaternion, or versor. The quaternions covered outside this appendix can be considered unit quaternions. The $\otimes$ will be usually omitted too. We've seen that quaternion algebra can be applied to scalars and vectors. Pure vector multiplication notation will remain $\mathbf{u}\cdot\mathbf{v}$ and $\mathbf{u}\times\mathbf{v}$ for clarification, despite these also being quaternion operations.

\section*{Quaternion Rotations}
A unit quaternion can be used to describe a rotation between bases. A vector described in unit basis $A$ would be notated $^A\mathbf{v}$. The quaternion $^Aq^B$ transforms $^A\mathbf{v}$ to $^B\mathbf{v}$ by the following formula:
\begin{equation}
  {^B\mathbf{v}} = \left({^Aq^B}\right)\left( {^A\mathbf{v}}\right)\left( {^Aq^{B*}}\right)
\end{equation}
By pre- and post-multiplying quaternion conjugates to invert the rotation $\left({^Aq^{B*}}\right)\left({^B\mathbf{v}}\right)\left({^Aq^B}\right)={^A\mathbf{v}}$, we see that the inverse of the rotation is just its conjugate. Since we are only talking about two reference frames in this paper: the inertial and the robot; rotating between the two can be described with a quaternion and its conjugate. Rotating from the inertial to the robot frame is chosen to be ${^Nq^R}=q$ and the robot to the inertial frame ${^Rq^N}=q^*$.

By Euler's rotation theorem, every rotation in 3-space can be described by a rotation axis and angle. For the transformation above, that quaternion is
\begin{equation}
  q = \exp\left(\frac{\theta}{2}\hat{\mathbf{u}}\right) = \cos\left(\frac{\theta}{2}\right) + \mathbf{u} \sin\left( \frac{\theta}{2}\right)
\end{equation}
where $\hat{\mathbf{u}}$ is the axis of rotation (scaled so that $\|q\|=1$) and $\theta$ is the rotated angle. Every rotation can be composed of small rotations when differentiated w.r.t. time gives:
\begin{equation}
  \dot{q} = \frac{1}{2} \dot{theta} \ \exp\left(\frac{\theta}{2} \hat{\mathbf{u}} \right)
\end{equation}

\section*{Quaternion Kinematics}
Integrating the quaternion into 

The quaternion is related to angular velocity by the following formula
\begin{equation}
  \dot{q} = \frace{1}{2} q \otimes \boldsymbol\omega
\end{equation}



Quaternion rotation from angular velocity
\begin{equation}
  \dot{q} = \frac{1}{2} q\boldsymbol\omega
\end{equation}

Linear interpolation
\begin{equation}
  q_{k+1} = \frac{1}{2}q_k\boldsymbol\omega\Delta t + q_k 
\end{equation}

Spherical interpolation?
\begin{equation}
  q_{k+1} = \exp\left(\frac{1}{2} \boldsymbol\omega\Delta t\right)q_k
\end{equation}


%{\appendix[Proof of Zonklar Equations]
%Use $\backslash${\tt{appendix}} if you have a single appendix:
% Do not use $\backslash${\tt{section}} anymore after $\backslash${\tt{appendix}}, only $\backslash${\tt{section*}}.
% If you have multiple appendixes use $\backslash${\tt{appendices}} then use $\backslash${\tt{section}} to start each appendix.
%You must declare a $\backslash${\tt{section}} before using any $\backslash${\tt{subsection}} or using $\backslash${\tt{label}} ($\backslash${\tt{appendices}} by itself starts a section numbered zero.)}

%{\appendices
%\section*{Proof of the First Zonklar Equation}
%Appendix one text goes here.
% You can choose not to have a title for an appendix if you want by leaving the argument blank
%\section*{Proof of the Second Zonklar Equation}
%Appendix two text goes here.}



% \begin{thebibliography}{1}
% \bibliographystyle{IEEEtran}


%\end{thebibliography}



% \vfill

\end{document}

